\documentclass[]{article}

\usepackage{
	amsmath, 
	amssymb,
	bera,
	parskip
}

\title{Assignment 2 ReadMe}
\author{
	Daniel Bok \\
	ESD \\
	1001049 \\
	daniel\_bok@mymail.sutd.edu.sg 
	\and
	Wong Yan Yee\\ 
	ISTD \\
	1001212 \\
	yanyee\_wong@mymail.sutd.edu.sg
	\and
	Clement Tan \\
	ESD \\
	1000948 \\
	clement\_tan@mymail.sutd.edu.sg
}
\date{\today}

\begin{document}
	
\maketitle
	
\section{Part 3 Derivation}

\subsection{Without Regularization}

\begin{align}
&\underset{b}{\arg\min} \quad \epsilon^2 \nonumber \\
\epsilon^2 &= \| Ab - c \|^2 \nonumber \\
&= (Ab - c)^T(Ab - c) \nonumber \\
&= (b^TA^T - c^T)(Ab - c) \nonumber \\
&= b^TA^TAb - b^TA^Tc - c^TAb + c^Tc \nonumber \\
&= b^TA^TAb - 2c^TAb + c^Tc \label{eqn:equation1}
\end{align}

We have  $b^TA^Tc = c^TAb$ because both of them are identical scalar variables. Differentiating Equation \ref{eqn:equation1}, we have

\begin{align}
\frac{\partial \epsilon^T\epsilon}{\partial b} 
	&= \frac{\partial b^TA^TAb - 2c^TAb + c^Tc}{\partial b} \nonumber \\
&= -2A^Tc + 2A^TAb \label{eqn:differentation1}\\
&\text{Set Equation \ref{eqn:differentation1} to 0} \nonumber \\ 
0 &= 2A^TAb - 2A^Tc \nonumber \\
b &= (A^TA)^{-1}A^Tc
\end{align}

\textbf{Notes}

By definition, $\frac{\partial b^TVb}{\partial b} = (V + V^T)b$. But since $A^TA$ is symmetric, Equation \ref{eqn:differentation1} reduces the term to $2A^TAb$.

\subsection{With Regularization}

With the regularization term, $\lambda\|b\|^2_2$, the minimization problem becomes

\begin{align}
&\underset{b}{\arg\min} \quad \epsilon^2 \nonumber \\
\epsilon^2 &= \| Ab - c \|^2 + \lambda \| b \|^2 \nonumber \\
&= (Ab - c)^T(Ab - c) + \lambda b^Tb \nonumber \\
&= (b^TA^T - c^T)(Ab - c) + \lambda b^Tb \nonumber \\
&= b^TA^TAb - b^TA^Tc - c^TAb + c^Tc + \lambda b^Tb \nonumber \\
&= b^TA^TAb - 2c^TAb + c^Tc + \lambda b^Tb \label{eqn:equation2}
\end{align}

Differentiating Equation \ref{eqn:equation2}, we have

\begin{align}
\frac{\partial \epsilon^T\epsilon}{\partial b} 
	&= \frac{\partial b^TA^TAb - 2c^TAb + c^Tc + \lambda b^Tb}
		{\partial b} \nonumber \\
&= -2A^Tc + 2A^TAb + \lambda b \label{eqn:differentation2}\\
&\text{Set Equation \ref{eqn:differentation2} to 0} \nonumber \\ 
0 &= 2A^TAb - 2A^Tc + \lambda b \nonumber \\
b &= (A^TA + \lambda I)^{-1}A^Tc \label{eqn:diff_final_2}
\end{align}

\textbf{Notes} 

While in Equation \ref{eqn:diff_final_2} we should have taken the term to be $0.5\lambda I$ to be precise, we have chosen not to do so as $\lambda$ is a constant hyper-parameter. We can thus allow $0.5$ to be absorbed into the hyper-parameter with no change to the consistency of Equation \ref{eqn:diff_final_2}.

\end{document}
